\bichapter{绪论}{Introduction}\label{chap:intro}
模板使用之初需输入论文信息,模板附带了~\verb|ezinfo|宏包,提供键值型接口~\verb|bjutset|接收输入的论文信息,每一个键都对应论文的一项信息。模板预设了常用键项,保存在模板一级目录下~\verb|ezinfo.cfg|配置文件内,示例文档~\verb|main.tex|导言区给出了输入键值的样例,形如:

{\linespread{1}\small
\begin{verbatim}
\bjutset {%
    clc            = {TU311},      % 中文图书分类号
    udc            = {004},        % 十进制分类号
    schoolcode     = {10005},      % 学校编号
    secretlevel    = {公开},       % 密级
    studentid      = {B20xxxxxxx}, % 学号
    % -- 中文信息
    ctitle         = {中文题目},
    cauthor        = {作者}, 
    cdiscipline    = {学科}, 
    cmajor         = {研究方向},
    cdegree        = {学位类型}, 
    csupervisor    = {导师},
    csupervstitle  = {导师职称},
    ccollege       = {学院},  
    cdate          = {中文日期},
    corganization  = {学位授予单位},
    % -- 英文信息
    etitle         = {English Title},
    edegree        = {Doctor of Engineering},
    emajor         = {Basketball},
    eauthor        = {YAO Ming},
    esupervisor    = {DU Xiuli},
    ecollege       = {College of Architecture and Civil Engineering},
    edate          = {\today}
}
\end{verbatim}}

模板还附带了双语目录宏包~\verb|bitoc|,提供了三组命令~\verb|\bichapter|、\verb|\bisection|、\verb|\bisubsection|用于定义双语标题。其使用格式为:
\begin{center}\small
\verb|\bichapter{<中文标题>}{<English Title>}|
\end{center}
输入的双语标题会自动列入双语目录中。

正文内容宜分章、节、条、款、项五级,前三级列入目录。各级标题格式示例如下。章标题,三号黑体居中。

\bisection{离离原上草}{li li yuan shang cao}
节标题,四号黑体顶格,正文另起一行,首行缩进两字符。

\bisubsection{一岁一枯荣}{yi sui yi ku rong}
条标题,小四号黑体顶格,正文另起一行,首行缩进两字符。

\subsubsection{野火烧不尽}
款、项标题小4号楷体,题序顶格书写,与标题间空一格,下面阐述内容在标题后空一格接排。

\subsubsection{春风吹又生}
正文小四号宋体。

\bisection{引用示例}{Cross reference examples}\label{sec:llysc}
模板中的文献处理使用~\verb|gbt7714|宏包,其文献格式排版遵循国标GB/T-7714(2015)。\verb|gbt7714|宏包的文献引用依赖于~\verb|natbib|宏包,且已默认加载,无需额外加载命令,引用命令也与~\verb|natbib|宏包一致,模板开启了压缩文献编号选项,如遇编号连续的多篇文献会自动使用连字符压缩序号,仅保留首尾序号。

模板使用~\verb|cleveref|~宏包实现标签的交叉引用,并已经对引用标签进行了中文适配,适配内容保存在模板一级目录下的~\verb|cleveref.cfg|~配置文件内,如有特殊需求可自行修改,配置文件内的设定会被模板自动加载。以下为引用示例。

文献普通引用\cite{同鸣2012},文献叙述引用\citet{Boutsidis2011},多组文献引用\cite{同鸣2012, Lee1999, Tang2013, Ding2006Orthogonal}。正文内容正文内容。章节引用\cref{chap:intro},章节引用\cref{sec:llysc},正文内容。公式引用\cref{eq:long},图引用\cref{fig:flow_chart},表引用\cref{tab:firstone}。

\bisubsection{公式示例}{Equations examples}
普通公式:
\begin{equation}
E=mc^2
\end{equation}

长公式:
\begin{multline}\label{eq:long}
\frac{1}{2}\Delta(f_{ij}f^{ij})=
2\left(\sum_{i<j}\chi_{ij}(\sigma_i-\sigma_j)^2+f^{ij}\nabla_j\nabla_i(\Delta f)+\right .\\
\left .\nabla_kf_{ij}\nabla^kf^{ij}+f^{ij}f^k\left[2\nabla_iR_{jk}-\nabla_kR_{ij}\right]\vphantom{\sum_{i<j}}\right)
\end{multline}

多行公式:
\begin{align}\label{eq:align}
&I(X_3;X_4)-I(X_3;X_4\mid{}X_1)-I(X_3;X_4\mid{}X_2)\nonumber\\
=&[I(X_3;X_4)-I(X_3;X_4\mid{}X_1)]-I(X_3;X_4\mid{}\tilde{X}_2)\\
=&I(X_1;X_3;X_4)-I(X_3;X_4\mid{}\tilde{X}_2)
\end{align}

方程组:
\begin{align}
\begin{cases}
\ u_{tt}(x,t)= b(t)\triangle u(x,t-4)&\\
\ \hspace{42pt}- q(x,t)f[u(x,t-3)]+te^{-t}\sin^2 x,  &  t \neq t_k; \\
\ u(x,t_k^+) - u(x,t_k^-) = c_k u(x,t_k), & k=1,2,3\ldots ;\\
\ u_{t}(x,t_k^+) - u_{t}(x,t_k^-) =c_k u_{t}(x,t_k), &
k=1,2,3\ldots\ .
\end{cases}
\end{align}

\bisubsection{图表示例}{Figures and tables}
插图应与文字紧密配合,文图相符,内容正确。选图要力求精练。

机械工程图:采用第一角投影法,严格按照GB4457--GB131-83《机械制图》标准规定。

电气图:图形符号、文字符号等应符合\cref{app:codes}~所列有关标准的规定。

流程图:原则上应采用结构化程序并正确运用流程框图。

对无规定符号的图形应采用该行业的常用画法。

每个图均应有图题(由图号和图名组成)。图号按章编排,如第1章第一个插图的图号为“图 1-1”等。图题置于图下,用中、英文两种文字居中书写,中文在上,要求用5号字。有图注或其它说明时应置于图题之上。图名在图号之后空一格排写。引用图应注明出处,在图题右上角加引用文献号 。图中若有分图时,分图题置于分图之下,分图号用a)、b)等表示。图中各部分说明应采用中文(引用的外文图除外)或数字项号,各项文字说明置于图题之上(有分图题者,置于分图题之上)。

\begin{figure}[htbp]
	\centering
	{\input{fig/flow_chart.tex}}
	\bicaption{绘制流程图效果}{Sample Flow Chart}
	\label{fig:flow_chart}
\end{figure}

插入表格。表的编排建议采用国际通行的三线表即顶线、底线和栏目线 注意:没有竖线 。其中顶线和底线为0.75pt粗线,栏目线为0.5pt细线,排版三线表必要时可加辅助线,三线表的组成要素包括:表序、表题、项目栏、表体、表注。表头设计应简单明了,尽量不用斜线。表头中可采用化学符号或物理量符号。

每个表格均应有表题(由表序和表名组成)。表序一般按章编排,如第1章第一个插表的序号为 “表 1-1”等。表序与表名之间空一格,表名中不允 许使用标点符号,表名后不加标点。表题置于表上,用中、英文两种文字居中排写,中文在上,要求用5号字。

全表如用同一单位,则将单位符号移至表头右上角,加圆括号。

表中数据应准确无误,书写清楚。数字空缺的格内加横线“—”(占2个数字宽度)。表内文字或数字上、下或左、右相同时,采用通栏处理方式,不允许用$ '' $、同上之类的写法。

表内文字说明,起行空一格、转行顶格、句末不加标点。
\begin{table}[!hpb]
	\centering
	\bicaption[指向一个表格的表目录索引]
	{一个颇为标准的三线表格\footnotemark[1]}
	{ATableExample}
	\label{tab:firstone}
	\begin{tabular}{@{}llr@{}}\toprule
		\multicolumn{2}{c}{Item}\\\cmidrule(r){1-2}
		Animal&Description&Price(\$)\\\midrule
		Gnat&pergram&13.65\\
		&each&0.01\\
		Gnu&stuffed&92.50\\
		Emu&stuffed&33.33\\
		Armadillo&frozen&8.99\\\bottomrule
	\end{tabular}
\end{table}
\footnotetext[1]{这个例子来自\href{http://www.ctan.org/tex-archive/macros/latex/contrib/booktabs/booktabs.pdf}{《Publication quality tables in LATEX》}(booktabs宏包的文档)。这也是一个在表格中使用脚注的例子,请留意与threeparttable实现的效果有何不同。}

下面一个是一个更复杂的表格,用threeparttable实现带有脚注的表格,如表\ref{tab:footnote}。

\newcolumntype{d}[1]{D{.}{.}{#1}}% or D{.}{,}{#1} or D{.}{\cdot}{#1}
\begin{table}[!htpb]
	\bicaption[出现在表目录的标题]
	{一个带有脚注的表格的例子}
	{A Table with footnotes}
	\label{tab:footnote}
	\centering
	\begin{threeparttable}[b]
		\begin{tabular}{ccd{4}cccc}
			\toprule
			\multirow{2}{6mm}{total}&\multicolumn{2}{c}{20\tnote{1}} & \multicolumn{2}{c}{40} &  \multicolumn{2}{c}{60}\\
			\cmidrule(lr){2-3}\cmidrule(lr){4-5}\cmidrule(lr){6-7}
			&www & \multicolumn{1}{c}{k} & www & k & www & k \\ % 使用说明符 d 的列会自动进入数学模式,使用 \multicolumn 对文字表头做特殊处理
			\midrule
			&$\underset{(2.12)}{4.22}$ & 120.0140\tnote{2} & 333.15 & 0.0411 & 444.99 & 0.1387 \\
			&168.6123 & 10.86 & 255.37 & 0.0353 & 376.14 & 0.1058 \\
			&6.761    & 0.007 & 235.37 & 0.0267 & 348.66 & 0.1010 \\
			\bottomrule
		\end{tabular}
		\begin{tablenotes}
			\item [1] the first note.% or \item [a]
			\item [2] the second note.% or \item [b]
		\end{tablenotes}
	\end{threeparttable}
\end{table}