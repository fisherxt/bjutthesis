\begin{cabstract}
	摘要是论文内容的简要陈述,是一篇具有独立性和完整性的短文。摘要应包括本论文的基本研究内容、研究方法、创造性成果及其理论与实际意义。
	
	摘要的字数(以汉字计),硕士学位论文一般为500~1000字,博士学位论文为1000~2000字,均以能将规定内容阐述清楚为原则。摘要页不需写出论文题目。
	
	关键词是供检索用的主题词条,应采用能覆盖论文主要内容的通用技术词条(参照相应的技术术语标准)。关键词一般列3~5个,按词条的外延层次从大到小排列。
	
	摘要中应排除本学科领域已成为常识的内容;切忌把应在引言中出现的内容写入摘要;一般也不要对论文内容作诠释和评论(尤其是自我评价)。
	
	摘要中不宜使用公式、图表,不标注引用文献编号。避免将摘要写成目录式的内容介绍。
	
	摘要结构要严谨、表达简明、语义确切。一般通用第三人称。建议采用“对……进行了研究”、“报告了……现状”、“进行了……调查”等记述方法标明学位论文的主题,不必使用“本文”、“作者”等作为主语。
	
	摘要题头应居中,然后书写摘要的正文部分。中文摘要正文内容及关键词用小4号宋体,多倍行距1.3。摘要正文之后隔一行顶格书写关键词。英文摘要和关键词应当与中文相同,英文摘要的实词应在300以上。
	
	关键字题头为小4号黑体。中文关键词字体为小4号宋体、英文摘要的关键词通常应用Times New Roman小4号字体。
	
	\ckeywords{关键词一;关键词二;关键词三;关键词四;关键词五}
\end{cabstract}

\begin{eabstract}
	\lipsum[1-4]
	\ekeywords{Keyword1, Keyword2, Keyword3, Keyword4, Keyword5}
\end{eabstract}